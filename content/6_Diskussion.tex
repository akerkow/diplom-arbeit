\subsection{Allgemeingültige Ergebnisse für für die Stationen entlang des Salzgradienten}

Zum Zeitpunkt der Untersuchungen entlang des Salzgradienten Ende Juni/ Anfang Juli gab es an allen 6 Stationen überall einen deutlichen Unterschied in der Bedeckung mit Makrophytobenthos und in deren Anteil an der Wassersäule zwischen den Untersuchungsgruppen. Ausgenommen hiervon ist die Spandowerhagener Wiek, an der kein vegetationsarmer Standort vorgefunden werden konnte. Die Deckung betrug an allen spärlich besiedelten Stellen maximal \unit{2,5}{\%} und das PVI maximal 0,3%. Aufgrund der sehr verschiedenen PVI- Werte und Deckungsgrade der dicht besiedelten Standorte war es interressant mithilfe von nichtparametrischen Regressionsanalysen zu untersuchen, ob und in welchem Maß die genannten Vegetationsparameter die mittlere Korngröße beinflussen. Da das Verhältnis von Bedeckungsgrad und PVI stark schwankt (in Vitte ist es beispielsweise wesentlich größer als in der Orther Bucht), wurde der Einfluss beider Vegetationsparameter auf das Sediment getrennt voneinander betrachtet. 

Es zeigt sich, dass es keinen Zusammenhang gibt, sowohl zwischen dem PVI und dem Median der Korngröße als auch zwischen der Bedeckung mit Phytobenthos und dem Median der Korngröße. Ebenso konnte kein Zusammenhang zwischen den Vegetations-Strukturparametern und dem Siltanteil der Sedimente gefunden werden.

Die Tatsache, dass sich neben der Vegetationsstruktur und Artenzusammensetzung die Standorte weiterhin in vielen anderen Aspekten voneinander unterscheiden macht es schwierig, allgemeingültige Schlüsse aus diesen Regressionsanalysen zu ziehen und auf den Zusammenhang bzw. auf eine Unabhängigkeit zwischen der Makrophytobenthosstruktur und dem Sediment zu schließen. Die Stationen unterscheiden sich nämlich auch in ihrer Wassertiefe voneinander. Diese wiederum hat einen direkten Einfluss auf die Fließgeschwindigkeit am Grund und damit auf die Bodenschubspannung. Aufgrund der unterschiedlichen geografischen Lage ist auch der Fetch aus der allgemeinen Hauptwindrichtung sowie der aktuelle Fetch während des Untersuchungszeitpunktes sehr unterschiedlich. Der Fetch ist nach \cite{laenen_1996} jedoch ein Parameter, der ebenfalls in die Berechnung der Bodenschubspannung eingeht, demzufolge müsste eine zusätzliche Abhängigkeit der Korngröße vom Fetch bestehen. Außerdem unterscheidet sich der Salzgehalt der Stationen voneinander, welcher wiederum nach \cite{schwenke_1995}  den Wuchs der Algen beeinträchtigt und für eine Veränderung der Artengemeinschaften im Ökosystem führt, da der Salzgehalt und seine Schwankungen eine von Organismen nur in bestimmten Toleranzbereichen ertragen wird. So könnte der Salzgehalt beispielsweise auch einen Einfluss auf die Benthosfauna haben.

In der schrittweise durchgeführten Regression unter Einbeziehung der Parameter Salinität, Wassertiefe, Fetch und PVI zeigt sich, dass eine Abhängigkeit der mittleren Korngröße von der den Faktoren Fetch und Salinität in Kombination die höchste Wahrscheinlichkeit aufweist. Es ist jedoch möglich, dass andere, unbekannte Faktoren wie zum Beispiel anthropogene lokale Einflüsse eine Rolle spielen.

Betrachtet man die Standorte und deren Untersuchungsgruppen separat voneinander, so lässt sich feststellen, dass  an 4 von 5 Stationen das Sediment in der dicht besiedelten Untersuchungsgruppe im Mittel feiner war, nämlich an den zwei Hiddenseer Standorten, im Salzhaff und in der Griebener Bucht. 




\subsection{Zusammenhänge zwischen Vegetation und Sediment in den Hiddenseer Boddengewässern}

Die Vegetation auf den Untersuchungsflächen in der Griebener Bucht und im Vitter Bodden repräsentiert gut die für die Hiddenseer Boddengewässer typischen Artenzusammensetzungen, verglichen mit Beobachtungen von \cite{flugge_2004}, \cite{kuenzenbach_1955} und \cite{muller_1961}.
\\
Im Vitter Bodden, in etwa \unit{80}{\centi\metre} Wassertiefe fanden sich dichte, lose auf dem Boden aufliegende Matten des Blasentanges \textit{Fucus vesiculosus} welcher in dieser Wassertiefe zwischen der Fährinsel und dem Hafen in Vitte das Vegetationsbild prägt und zusammen mit Begleitarten wie \textit{Furcellria fastigiata} \textit{Potamogeton pectinatus}, \textit{Myriophyllum spicatum} und \textit{Chorda filum} vorkommt. Bei \unit{100}{\%}. Das Laichkraut und das Tausendblatt sind sehr häufige und typische Vertreter der Bodden, sie konnte ihr Vorkommen in der Griebener Bucht und im Vitter Bodden ab etwa \unit{50}{\centi\metre} Wassertiefe oft beobachtet werden. Für \textit{Myriophyllum spicatum}, nicht jedoch für \textit{Potamogeton pectinatus} fand \cite{flugge_2004} in der Griebener Bucht eine positive Korrelation mit der Wassertiefe. 
\cite{muller_1961}  beschreibt \textit{Potamogeton pectinatus} als dominante bestandsbildende Art in \unit{1,5 bis 4}{\metre} Wassertiefe und grenzt dies als Zone zu Beständen aus \textit{Chara baltica} und \textit{Fucus vesiculosus f. filiformis} ab. Die zuletzt genannten Arten kommen nach  \cite{muller_1961} in einer Wassertiefe von \unit{0,5 bis 2,5}{\metre} vor. 

Obwohl das Phytobenthos des vegetationsdominierten Standortes in Vitte nur einen Anteil von \unit{8-17}{\%} an der Wassersäule hat, ist die Biomasse auf den Flächen mit im Mittel \unit{890}{\gram\per\metre\suqared} (Trockensubstanz) im August extrem hoch. Auf gleich großen Ernteflächen im Golf von Riga fand \cite{martin_1999} ebenfalls sehr hohe Biomassewerte in festsitzenden \textit{Fucus vesiculosus} -Beständen. Zwischen Ainazhi und Kabli in \unit{1,5}{\metre} Tiefe fand er die höchsten Biomassewerte im Golf von Riga von \unit{2700}{\gram\per\metre\suqared} (Feuchtgewicht). Grund für die hohen Biomassewerte könnten die Derbheit der Thalli und das große Masse-Volumen-Verhältnis sein, denn die Individuen waren annähernd kugelförmig mit einer in sich hohen Dichte. Die Flächen waren zudem zu \unit{100}{\%} bedeckt und die Braunalgen zu solch dichten Matten zusammengeschoben, dass  der Wasseraustausch zwischen den Wasserschichten innerhalb des Algenbettes und darüber erschwert war, sodass sich nach eigenen Beobachtungen von Mai bis Anfang Juni eine deutliche Thermokline feststellen ließ (in gleicher Wassertiefe ohne dichte \textit{Fucus vesiculosus f. balticus} -Bestände wurde diese Thermokline nicht festgestellt). 

Sowohl die Biomasse als auch das PVI veränderten sich in der Untersuchungsgruppe. Von Juni bis August stieg der Anteil der Pflanzen in der Wassersäule um \unit{7,3}{\%} und auch in den 2 Biomassekartierungen im Juni und im August wurde ein mittlerer Biomassezuwachs von \unit{90}{\gram\per\metre\squared} (Trockengewicht) festgestellt. 
In dem von Feinsand und Mittelsand geprägten Sediment konnte jedoch keine Veränderung der mittleren Korngröße festgestellt werden. Ebenso änderte sich der Anteil der \unit{<63}{$\mu$\metre} Fraktion und der organische Gehalt nicht im Jahresverlauf. 
\\
Die spärlich besiedelte Vergleichsgruppe des Vitter Standortes befand sich etwa \unit{300}{\metre} weiter östlich auf einer Sandbank, das Wasser war hier mit einer mittleren Tiefe von \unit{62}{\centi\metre} etwa \unit{20}{\centi\metre} flacher. An einigen Untersuchungstagen mit Wasserständen unter dem Mittel wurden hier teilweise nur \unit{30-40}{\centi\metre} Tiefe gemessen. Erst ab Juli befanden sich hier lichte Characeen-Bestände von \textit{Chara canescens} und \textit{Chara baltica}, typische Brackwassercharaceen, die auch stark schwankende Salzgehalte vertragen \citep{blindow_2003, blumel_2003}. 

\textit{Chara canescens} kommt hauptsächlich in Flachwasserbereichen unter \unit{50}{\centi\metre}  bis maximal \unit{1}{\metre} Wassertiefeiefe \citep{blindow_2003} vor und überwintert hauptsächlich in Form von Oosporen \cite{wahlstedt_1862}. Auch \textit{Chara baltica} ist typisch für Flachwasserbeiche, sowohl in geschützten Buchten als auch an exponierten Ufern \citep{blumel_2003}. Die Art überwintert hauptsächlich in Form von Bulbillen, selten als Pflanze oder in Form von Oosporen \cite{blumel_2003}. Während \cite{flugge_2004} beide Characeenarten  bereits Ende Mai kartieren konnte, waren sie im Untersuchungsjahr 2013 erst ab Juni in den Hiddenseer Bodden zu beobachten. Daraus lässt sich schließen, dass sich die Characeen-Vegetation aufgrund des lang anhaltenden Winters allgemein erst recht spät im Jahr entwickelt hat.
Neben den Characeen fanden sich ab August auch \textit{Ruppia cirrhosa}, \textit{Potamogeton pectinatus} und \textit{Myriophyllum spicatum} auf den Flächen im spärlich besiedelten Bereich des Vitter Boddens. Die Tatsache, dass sie erst so spät im Jahr kartiert wurden, lässt darauf schließen, dass sie im Gegensatz zu den Pflanzen im dicht besiedelten Bereich aus Samen ausgekeimt sind und nicht als Sporophyt überwintert haben.

Wähhrend sich die Bedeckung und das PVI an dieser Untersuchungsgruppe von unit{0}{\%} auf \unit{5 bzw. 0,6}{\%} erhöht haben, ändert sich die Korngrößenverteilung, der Median der Korngröße der mittel- und feinsanddominierten Sedimente nicht im Verlauf der Wachstumssaison.

Im Vergleich beider Untersuchungsgruppen in Vitte wurde festgestellt, dass das Sediment in der östlich gelegenen vegetationsarmen Gruppe im Mittel etwas gröber ist, der Anteil des organischen Gehaltes jedoch identisch. 

Sowohl für die Griebener Untersuchungsgebiete als auch für die Vitter Untersuchungsgebiete kann gesagt werden, dass sich die Vegetationsstruktur (mittlere Wuchshöhen und PVI) im Jahresverlauf ändern und dass es einen Wechsel an Dominanzen verschiedener Arten im Verlauf der Wachstumssaisong gibt. Die Sedimentstruktur jedoch ändern sich nicht im Verlauf der Wachstumsperiode. Dies hängt damit zusammen, dass andere Faktoren wie Wassertiefe und Fetch neben weiteren nicht in Betracht bezogenen Faktoren einen Einfluss auf die Korngrößenverteilung haben können. 


\subsection{Allgemeingültige Ergebnisse für für die Stationen entlang des Salzgradienten}
