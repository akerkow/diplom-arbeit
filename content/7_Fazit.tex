Entlang des Salzgradienten wurden 6 Standorte mit typischen Bodden-Vegetationsgesellschaften untersucht, dabei unterschied sich ihre Vegetation in Hinsicht auf Artenzusammensetzung, ihren Anteil der Blätter und Thalli an der Wassersäule und ihre Bedeckung deutlich voneinander. Es gab Grundrasen-dominierte Bestände mit geringen PVI-Werten und Dominanzen der zierlichen Arten \textit{Ruppia cirrhosa} und \textit{Zannichellia palustris}, Bestände mit hohen Deckungsgraden und hohen PVI-Werten mit dominanten Arten wie Seegras (\textit{Zostera marina}),  Parvopotamiden und \textit{Myriophyllum spicatum} sowie \textit{Fucus vesiculosus f. balticus}- dominierte Bestände mithoher Deckung, extrem hohen Biomassewerten, jedoch einem Geringen Anteil des Phytobenthos an der wassersäule.
Es zeigte sich, 

Analysen in Strömungskanälen sinnvoll, um Pfl-Boden, boden-Pflanzenwechselwirkung zu klären!

strömungsversuche mit den typischen Boddenarten! Fucus, Potamogeton, Seegras, Myriophyllum, Ruppia, Zannichellia
= Einfluss der Pfl.arten auf strömung + sediment

trockensiebung oder absetzversuche von feinmaterial für genauere Analysen