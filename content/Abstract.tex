 The following study is part of the 'BACOSA' cooperate project (Baltic Coastal System Analysis and Status Evaluation). The projects aim is to comprehensively understand the internal nutrient flow in shallow waters of the Baltic Sea and, according to this, the importance of macrophyte beds and sediments. 
 
 Therefore, vegetation structure and species composition, sediment structure (grain size distribution and median grain size as well as organic matter content of sediments) were analysed at 6 sites along the German coast (Geltinger Bucht, Orther Bucht, Salzhaff, Hiddensee-Bodden and Spandowerhagener Wiek). Two groups where distinguished at each side: one with dense vegetation (at least \unit{50}{\%} coverage) and one with sparse vegetation (\unit{2}{\%} maximum coverage) and there where 5 replicates for each group. 
 
Using Man-Whithney-Tests it was examined if there are differences between sediment structures and total vegetation (total cover and the amount of plants infested in the water column, PVI). Additionally, vegetation and sediment were sampled at 2 sites in the Hiddensee-Bodden (Vitter Bodden and Griebener Bucht) for 5 times during the whole plant growing season. Differences of each parameter in time where determined with Kruskal-Wallis- and multiple comparism tests. 

The results show that there are 3 types of vegetation structures: Firstly, one with \unit{40 to 60}{\%} Coverage but low PVI because of the dominance of small meadow forming species like \textit{Ruppia cirrhosa} and \textit{Zannichellia palistris}. Secondly one type with \unit{60 to 70}{\%} coverage combinated with a high amount of PVI and hight growing species like Seaweed (\textit{Zostera marina}). Finally, there is a vegetation type with very high total coverage of \unit{100}{\%} but low PVI caused by beds mainly consisting of small macroalgeae (\textit{Fucus vesiculosus f. balticus}). 

The amount of organic matter in the sediment is higher in dense vegetation beds with exception to the \textit{Zannichellia palustris} dominated Orther Bucht and there is a positive correlation between total coverage and organic matter in the sediment as well as between PVI and organic matter content. 
No relationships between vegetation and median grain size and between vegetation and the percentage of silt and very fine sand could be found. While vegetation structure, grain size distribution and organic content of the sediment did not change in the Vitter Bodden during the growing season, vegetation structure in the Griebener Bucht increased at the same time with median grain size.







