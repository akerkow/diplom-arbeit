Hiermit möchte ich mich bei allen bedanken, die mich bei meiner Diplomarbeit unterstützt haben.

Mein ganz besonderer Dank geht an Jutta Meyer, die mit unermüdlicher Ausdauer fast alle Schnorcheltouren bei jedem Wind und Wetter begleitet hat, die im Labor etliche Stunden Sedimente mit mir gesiebt und Biomasseproben ausgewaschen hat und die für jedes Problemchen immer ein offenes Ohr hatte. Ebenso herzlich möchte ich mich bei Milena Kafka, Caroline Lindner und Bozena Nawka für die gute Zusammenarbeit und bei Irmgard Blindow für die engagierte und intensive Betreuung bedanken.

Bei den Mitarbeitern des Nationalparkamtes möchte ich mich bedanken, dass sie diese Studie überhaupt ermöglicht und die Erlaubnis erteilt hat, für wissenschaftliche Zwecke im begrenzten Umfang die Schutzgebietszonen bei Hiddensee zu befahren und zu beproben.

Desweiteren möchte ich mich ganz herzlich bei Sven Dahlke bedanken, der mit viel Begeisterung, Fachkenntnis und Blick fürs Detail wichtige Probenahmeutensilien in Eigenbau angefertigt hat, der mich in die Unterwasserfotografie eingeführt und mir seine Unterwasserkamera zur Verfügung gestellt hat.

Auch bei Helmut Ehmke und Lothar Spengler möchte ich mich bedanken für die Anfertigung von Probenahmegeräten, für das Beibringen des Motorbootfahrens und die Begleitung auf See. Ebenso gilt mein Dank Wolfgang und Gerlinde Zenke, die sich um viel Organisatorisches und alles rund um das Labor gekümmert haben.

Bei Franziska Bitschofsky möchte ich mich für die Bereitstellung ihrer Analyseergebnisse des Sedimentes entlang des Salzgradienten bedanken und bei Maike Piepho für die Organisation der Unterkünfte für unsere Salzgradienten-Tour.

Zu guter Letzt geht ein ganz herzliches Dankeschön an meinen Bruder Daniel Kerkow, der mir für Fragen rund um das Anfertigen von Karten und das Schreiben in Latex zur Seite stand und an meine Eltern für die Unterstützung während des Schreibens der Diplomarbeit.