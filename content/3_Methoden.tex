\subsection{Vegetationskartierung}

Um die Vegetation (und weitere Parameter) zu untersuchen, wurden die Plots mit einem motorisieren Ruderboot von der Lee-Seite angefahren, wobei unmittelbar in der Nähe eines Plots der Motor möglichst nicht in Betrieb genommen wurde. Der Anker wurde so geworfen, dass er weit genug entfernt war, um kein Sediment über der Fläche aufzuwirbeln und dass das Boot genau an der Lee-Seite der Fläche zum Stehen kam. Dann wurde (nach der Lichtmessung und der Wasserprobeentnahme für weitere Analysen) ein absenkbarer Aluminium-Vegetationsrahmen der Größe zwei mal zwei Meter so abgelegt, dass der Zementstein der Schwimmboje die Süd-West-Ecke des Plots einnahm. \\
Nun wurde per Schnorcheln die Vegetationsbedeckung insgesamt und für jede vorkommende Pflanzen-und Großalgenart geschätzt. Auch fädige Algen, die im Frühjahr bis in den Juni hinein große deckende Matten gebildet hatten, wurden mit berücksichtigt, jedoch wurde dieses Geflecht unterschiedlichster Gattungen und Arten nicht genauer bestimmt. \\ 
Zur Identifizierung der nicht ohne Hilfsmittel erkennbaren Arten, insbesondere der Characeen-Arten, wurden anfänglich jeweils gleich aussehende Exemplare aus dem Plot-angrenzenden Gebiet entnommen und im Labor mit dem Binokular bestimmt, später im Jahr konnten sie bei guten Sichtverhältnissen ohne Hilfsmittel bestimmt werden. Die Gefäßpflanzen wurden bestimmt nach \cite{van_de_weyer_2007} und \cite{rothmaler_2005} und die Algen nach \cite{pankow_1971}.\\


\begin{table}[h]{\textwidth}
\centering
\caption{Skala für die Schätzung der Makrophytenbedeckung}
\begin{tabular}{lll}
\toprule
Code	&& Translation\\
\midrule
0,5	 	&& \unit{<1}{\%} \\
2 		&& \unit{1-4}{\%} \\
5 		&& \unit{5}{\%}\\
\midrule
10 		&& \unit{6-14}{\%}\\
20 		&& \unit{15-24}{\%}\\
30 		&& \unit{25-34}{\%}\\
40		&& \unit{35-44}{\%}\\
50		&& \unit{45-54}{\%}\\
\midrule
60		&& \unit{55-64}{\%}\\
70		&& \unit{65-74}{\%}\\
80		&& \unit{75-84}{\%}\\
90		&& \unit{85-94}{\%}\\
100		&& \unit{95-100}{\%}\\
\bottomrule
\end{tabular}
\label{Deckung}
\end{table}



Die Schätzung der Deckungsgrade erfolgte nach einer Skala, die im Bereich der unteren Deckungsgrade an die Braun-Blanquet-Skala angepasst, jedoch im Bereich größerer Deckungen eine höhere Genauigkeit aufweist (Vgl. Tabelle \ref{Deckung}).\\
Außerdem wurde eine Höhenstufenkartierung an jedem Plot durchgeführt. Dabei wurde die Deckung der Vegetation insgesamt und für jede vorkommende Art in unterschiedlichen Abständen  vom Grund geschätzt. Dabei wurden auf den Wuchshöhen \unit{5}{\centi\metre}; \unit{10}{\centi\metre} und alle weiteren zehn Zentimeter bis zur Oberfläche kartiert.\\






\subsection{PV und PVI}

Der Anteil des Pflanzenvolumens am Gesamtwasservolumen (Plant Volume Infested, PVI), wird nach \cite{jeppesen_1998, schriver_1995, canfield_1984} berechnet als 

\begin{equation*}
\frac{\text{Mittlere Wuchshöhe} * \text{Deckung}}{\text{Wassertiefe}}.
\end{equation*}

Die mittlere Wuchshöhe, multipliziert mit der Deckung, wird auch als 'Area specific plant volume', PV, bezeichnet \citep{jeppesen_1998}.\\
In dieser Studie wurde ein genaueres Verfahren angewendet, in dem die Deckungen aller Höhenstufen (vgl. Vegetationskartierung, Höhenstufenkartierung) bei der Berechnung berücksichtigt wurden:


\begin{align*}
 PV &=\sum_{i=a}^n \frac{H_i * C_i}{100} & H_i &=\text{Länge einer jeden Höhenschicht}\\ 
 & & C_i &=\text{Deckung auf dieser Höhenschicht}\\
 & & a-n &=\text{Höhehstufen vom Boden}\\
 & &     &\text{\quad bis zur Wasseroberfläche}\\
 PVI &=\frac{PV}{\text{mittlere Wassertiefe}}.\\
\end{align*}



\subsection {Mittlere Wassertiefe}

Die Wassertiefen wurden an jedem Plot mit dem Zollstock gemessen und zusammen mit der Uhrzeit notiert. Später wurde der mittlere Wasserstand und die Abweichung davon zum Zeitpunkt der Messung für die nächstgelegene Pegelmessstation recherchiert. Mit dieser Information konnte  der mittlere Wasserstand für jeden Plot berechnet werden.\\
Die Pegelmessstationen für die jeweiligen Standorte waren Kappeln für die Geltinger Bucht, Heiligenhafen für die Orther Bucht, Timmendorf für das Salzhaff, Kloster für die Hiddenseeer Standorte und Stahlbrode für die Spandowerhagener Wiek. Die minutengenauen Wasserstände hielten die Wasser- und Schifffahrtsämter von Stralsund und Lübeck bereit. Die Mittelwasserstände stammen von der Webseite Pegelonline des \cite{wasser-_und_schifffahrtsverwaltung_des_bundes_2013}.



\subsection{Biomasse} 


Um die Biomasse zu untersuchen, wurde ein \unit{20}{\centi\metre} hoher quadratischer Stahlrahmen mit einer Kantenlänge von \unit{50}{\centi\metre} angefertigt. Dieser wurde in ungefähr einem halben Meter Abstand zum Hauptplotuntersuchungs-Plot auf dem Boden abgesetzt und leicht ins Sediment eingedrückt. Bei der Auswahl der benachbarten kleineren Fläche, nachfolgend als Miniplot bezeichnet, wurde darauf geachtet, dass sich das Vegetationsbild  möglichst wenig von dem im Hauptuntersuchungs-Plot unterscheidet.\\
Die Biomasse wurde schnorchelnd mit einer Harke abgeerntet und in einen gewöhnlichen Angelkäscher gefüllt, wobei der Käscher ständig in einer leicht kreisenden Bewegung unter Wasser gehalten wurde, sodass die Biomasse nicht entweichen konnte. Um sicher zu gehen, dass keine Biomasse übersehen wurde, wurde zum Schluss noch einmal ein paar Minuten abgewartet, bis sich die Trübung durch das aufgewirbelte Sediment gelegt hatte und eventuell noch einmal nachgeerntet. Durch die harkende Methode wurde nicht nur die oberirdische sondern auch die Wurzelbiomasse mit abgeerntet und ging in die Analyse mit ein.\\
Für den Transport wurde die Biomasse in stabile Tüten verpackt und im Labor möglichst rasch bearbeitet. Hierfür wurde sie gewaschen und Steine und Muscheln herausgesammelt. Anschließend wurde sie bei \unit{105}{\celsius} so lange getrocknet, bis kein Gewichtsverlust durch weiteres trocknen mehr beobachtet werden konnte. Anschließend wurden die Proben im Exsikkator abkühlen lassen und gewogen. \\
Für die Bestimmung des aschfreien Trockengewichtes wurden jeweils \unit{30}{\gram} aus den getrockneten Proben in Aluminium-Schalen gefüllt, die vorher bei \unit{500}{\celsius} erhitzt wurden, im Exsikkator abgekühlt sind und danach gewogen wurden. Proben aus Plots mit einer sehr geringen Menge Biomasse wurden von vornherein in Schalen gefüllt, die auf diese Weise behandelt wurden. Die Subproben wurden dann bei \unit{500}{\celsius} drei Stunden im Ofen verbrannt. Die lineare Anheizzeit betrug 3 Stunden. Dann wurden die Proben wieder im Exsikkator abkühlen lassen und erneut eingewogen. Das prozentuale aschfreie Trockengewicht [AFDW (\%)]wurde dann berechnet als:

\begin{align*}
 AFDW(\%)&=\frac{100(B_d-B_b)}{B_d} & B_b &=\text{Gewicht der Biomasse}\\ 
 & &     &\text{\quad gebrannt bei 500\celsius}\\
 & & B_d &=\text{Gewicht der Biomasse}\\
 & &     &\text{\quad getrocknet bei 105\celsius}\\
\end{align*}




\subsection{Sediment}
\subsubsection{Analytik in Gelände und Labor}
Die Sedimententnahme unmittelbar neben jedem Plot erfolgte mittels eines Stechzylinders mit einem Innendurchmesser von \unit{10}{\centi\metre}, welcher mit Hilfe eines langen Carbon-Rohres vom Boot aus in das Sediment gestoßen wurde. Mit Hilfe einer am Rohr angebrachten, handgefertigten Schelle wurde der Zylinder am Rohr angebracht. Anschließend wurde er mit einem Stopfen mit Rücklassventil oben verschlossen, circa \unit{10}{\centi\metre} lotrecht ins Sediment geschoben und an die Oberfläche befördert, wobei das Rücklassventil beim Herausnehmen aus dem Wasser geschlossen gehalten und der Sedimentkern zusätzlich von unten mit einem Stopfen gesichert wurde. Auf dem Boot wurde zuerst der obere Stopfen abgenommen und das im Zylinder überstehende Wasser mit einem Schlauch von circa \unit{5}{\milli\metre} Innendurchmesser abgesaugt. Danach wurde der untere Stopfen entfernt und der Zylinder auf einem Ständer mit Gummistempel platziert. Das Sediment wurde mit dem Stempel aus dem Zylinder herausgedrückt und die obersten \unit{2}{\centi\metre} mit einem Spatel abgenommen und zur weiteren Untersuchung im Labor in feste Tüten verpackt.\\
Im Labor wurde jede Sedimentprobe zu einer Mischprobe verarbeitet und sichtbare Tier- und Pflanzenteile herausgesammelt. Für die Bestimmung des Wassergehaltes wurden Aluminiumschälchen vorbereitet. Für 2 Stunden wurden sie bei \unit{500}{\celsius} in den Ofen und danach in den Exsikkator zum Abkühlen gestellt. Anschließend wurden sie mit einer Feinwaage gewogen und das Gewicht notiert.\\
Aus den Mischproben wurden mit einer manuell vorn eingekürzten Spritze \unit{10}{\milli\litre} Probenmaterial entnommen und in die Aluminiumschälchen gefüllt. Danach wurde das Sediment 12 Stunden bei \unit{105}{\celsius} getrocknet, im Exikkator abgekühlt und erneut eingewogen. Anhand des Gewichtsunterschiedes der Probe vor und nach dem Trocknen konnte der prozentuale Wassergehalt ermittelt werden.\\
Um den organischen Gehalt in Form des aschfreien Trockengewichtes zu untersuchen, wurde das getrocknete Sediment aus den \unit{10}{\milli\litre}-Proben 12 Stunden bei \unit{500}{\celsius} in den Ofen gestellt, erneut im Exsikkator abgekühlt und eingewogen. Anschließend wurde der Anteil des Glühverlustes am Trockengewicht berechnet. \\
Für die Korngrößenanalyse wurden weitere \unit{100}{\gram} des frischen Sedimentes abgewogen und nass gesiebt. Dafür wurde eine Siebkaskade mit mittleren Korngrößendurchlässen von \unit{1000}{\micro\metre}, \unit{500}{\micro\metre}, \unit{250}{\micro\metre}, \unit{125}{\micro\metre} und \unit{63}{\micro\metre} verwendet. Der Anteil einer jeden Kornfraktion wurde in bei \unit{105}{\celsius} getrocknete  und danach eingewogene Aluminiumschälchen gefüllt und ebenso wie die \unit{5}{\milli\litre}-Proben mindestens 24 Stunden bei \unit{105}{\celsius} getrocknet und nach dem Abkühlen gewogen. Durch Aufaddieren aller Gewichte der getrockneten Korngrößen abzüglich des prozentualen Wassergehaltes konnte geschlussfolgert werden, wie groß der Anteil der Sedimentfraktion unter \unit{63}{\micro\metre} Durchmesser war. Mit Hilfe dieser Werte konnten die Gewichte aller gröberen Korngrößenfraktionen anteilig zur Summe der Gewichte aller Korngrößen berechnet werden.\\
Für die weitere statistische Auswertung wurden alle Korngrößenklassen mit dem negativen Logarithmus zur Basis 2 in $ \phi $-Werte umgerechnet. 
Zur Beschreibung der Korngrößenverteilungen wurden für jeden Plot an jedem Untersuchungstag Median, Mittelwert, Sortierung und Schiefe ausgerechnet und die Werte der jeweiligen Replikate für jedes Datum gemittelt.\\
\\
Die Sedimentproben entlang des Salzgradienten wurden von Franziska Bitschofsky (BACOSA-Doktorandin der Universität Rostock) aus den Untersuchungsflächen entnommen und bearbeitet. Dabei wurde jeweils nur ein Sedimentkern pro Standort für die Korngrößenanalyse aus dem dichtbewachsenen und aus dem spärlich bewachsenen Bereich entnommen. Aus jedem Kern wurden im Labor 3 Messparallelen untersucht. Die Nasssiebung, das Trocknen und Verbrennen erfolgten auf die gleiche Art wie bei den Hiddenseer Langzeitstandorten.


 
\subsubsection{Berechnungen der Kennwerte}


\begin{align*}
 AFDW(\%)&=\frac{100(S_d-S_b)}{S_d} & AFDW&=\text{Aschfreies Trockengewicht}\\ 
 & & S_b &=\text{Gewicht des Sedimentes}\\
 & &     &\text{\quad gebrannt bei 500\celsius}\\
 & & S_d &=\text{Gewicht des Sedimentes}\\
 & &     &\text{\quad getrocknet bei 105\celsius}\\
 L(\%)&=\frac{100(S_w-S_d)}{S_w} & L&=\text{Wassergehalt}\\
 & & S_w &=\text{Gewicht des feuchten Sedimentes}\\
 \\
 S_{d_i} (\%)&=\frac{100*S_d__i}{\sum_{k=a}^f S_d__k} & i&=\text{Gewicht einer Korngrößenfraktion}\\
 & & &\text{\quad der Kategorien a-f}\\
 S_{b_i} (\%)&=\frac{100*S_b__i}{\sum_{k=a}^f S_b__k}\\
 \\
 S_{d_{<63}} (\%) &=\frac{M-L(\%)}{100}-\sum_{k=a}^e S_d__k & M&=\text{Gesamteinwaage Feuchtprobe}\\
 \\
 \phi &=-log_2 \frac{d}{d_0} & d&=\text{Korndurchmesser}\\
 \\
 \tilde{x} &=\phi_{50} & \tilde{x}&= \text{Median}\\
 \\
 \bar{x}&=\frac{\phi_{16} +\phi_{50} + \phi_{84}}{3} & \bar{x} &= \text{Mittelwert} \\
 \\
 So &=\frac{\phi_{84}-\phi_{16}}{4} + \frac{\phi_{95}-\phi_5}{6,6} & So &= \text{Sortierung}\\             
 \\
 Sk &=\frac{\phi_{16}+\phi_{84}-2*\phi_{50}}{2*(\phi_{95}-\phi_{5})} + \frac{\phi_{5}+\phi_{95}-2*\phi_{50}}{2*(\phi_{84}-   \phi_{16})} & Sk &= \text{Schiefe} 
\end{align*}
\\
\\
\\
\\

\subsection{Datenanalyse und Kartenerstellung}

Dateneingabe und Berechnungen erfolgten in Microsoft Office Excel 2007. Histogramme, Boxplots und Balkendiagramme wurden mit IBM SPSS Statistics 20 ertsellt. Auch der Wilcoxon-Mann-Whitney-Test und der Kruskal-Wallis-Test wurden damit berechnet. Für multiple Vergleiche mit dem Dunn's-Test wurde das Programm GraphPad Prism 6 benutzt. Die Visualisierung der Regressionsmatritzen und die Bestimmung der besten multiplen Regressionsmodelle wurde mit Minitab 16 durchgeführt. Die Erstellung der Karten erfolgte mit QGIS 2.0.1. 

