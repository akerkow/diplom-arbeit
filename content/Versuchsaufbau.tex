\subsection{Hypothesen}




\subsection{Versuchsaufbau}

Im Hauptuntersuchungsgebiet, in den Boddengewässern östlich Hiddensees gelegen, sollte eine möglichst umfassende Studie der Zusammenhänge vieler Einzelparameter stattfinden und der Einfluss der Zeit mitberücksichtigt werden. Hier wurden jeweils an zwei Standorten fünf mal der Wuchs der Vegetation im Verlauf der Wachstumssaison von Juni bis September untersucht, davon zwei mal zusätzlich die Biomasse. Das Sediment wurde drei mal im Verlaufe der Saison untersucht. Parallel erfolgten Studien zu Strömungsverhältnissen, gelöstem Sediment in der Wassersäule und zu Phyto- und Zooplankton, ebenfalls fünf mal im Jahr an den gleichen Standorten. \\
Die Studie erfolgte in Gemeinschaftsarbeit von fünf Wissenschaftlerinnen, dabei wurden die Probenahmetermine für verschiedene Untersuchungen möglichst zeitnah miteinander verbunden, sodass bei der Auswertung des Datenmaterials aufeinander Bezug genommen werden konnte(Die genauen Termine sind in Tabelle \ref{Probenahmetermine} notiert.). Für die Befahrung und Beprobung der Standorte innerhalb des Nationalparkes lag eine Genehmigung der Aufsichtsbehörde des zuständigen Nationalparkamtes vor. \\
Zusätzlich zu den Hiddenseer Bodden wurden einmalig im Juni und Juli an 4 Standorten entlang des Salzgradienten in flachen Buchten und Boddengewässern der Ostsee  in Zusammenarbeit mit weiteren Wissenschaftlern der Universitäten Rostock und Kiel Vegetation, Sediment, Phytoplankton und suspendiertes Material untersucht(Probenahmetermine siehe Tabelle \ref{Salztabelle}.). \\
\\
An jedem Standort wurde zwischen 2 Gruppen unterschieden: eine mit dichter Vegetation (über 50\% Deckung) und eine ohne oder nur mit spärlicher Vegetation (nicht mehr als 2\% Deckung zu Beginn der Studie). In jeder Gruppe gab es fünf Messparallelen. Diese waren Plots der Größe zwei mal zwei Meter, die circa acht bis zwölf Meter voneinander entfernt lagen und die jeweils homogen in ihrem Vegetationsbild waren. Um diese Flächen im Jahresverlauf untersuchen zu können, waren sie jeweils an ihrer Südwest-Ecke mit einer nummerierten Schwimmboje markiert.\\ 
Die Gruppen selbst sollten auch möglichst nah aneinander und in der gleichen Wassertiefe liegen, wobei sich die Umsetzung dieser Kriterien in der Praxis als schwierig herausgestellt hat. Um einen deutlichen Unterschied in der Vegetation zu bekommen, mussten sie circa \unit{50-300}{\metre} voneinander entfernt liegen, was zum Beispiel in der \unit{54}{\metre} schmalen Griebener Bucht bedeutet, dass sich die Gruppen auf unterschiedlichen Seiten der Bucht befanden. 




\begin{table}[h]{\textwidth}
\caption{Übersicht Untersuchungen und Probenahmetermine}
\begin{tabular}{lllll}
\toprule
Nr. &  Date & Vitte & Date & Grieben\\
\midrule
\multirow{3}{*}{1} & Jun 04 & Vegetation & Jun 09 & Vegetation \\
 & Jun 05 & Susp. Material, Plankton & Jun 11 & Susp. Material, Plankton \\
 & Jun 05 & Sediment & Jun 11 & Sediment \\
\midrule
\multirow{4}{*}{2} & Jul 03 & Vegetation & Jul 05 & Vegetation \\
 & Jul 03 & Susp. Material, Plankton & Jul 05 & Susp. Material, Plankton \\
 & Jul 06 & Biomass  & Jul 05 & Biomass \\
 & Jul 05 & Sediment (F. Bitschofsky)\\
 \midrule
 \multirow{4}{*}{3} & Aug 01 & Susp. Material, Plankton & Jul 30 & Susp. Material, Plankton\\
 & Aug 01 & Sediment & Jul 30 & Sediment\\
 & Aug 06 & Vegetation & Aug 07 & Vegetation\\
 & Aug 06 & Biomass & Aug 07 & Biomass\\
 \midrule
 \multirow{3}{*}{4} & Aug 19 & Susp. Material, Plankton &  Aug 17 & Susp. Material, Plankton\\
 & Aug 19 & Sediment & Aug 17 & Sediment\\
 & Aug 21 & Vegetation & Aug 21 & Vegetation\\
 \midrule
 \multirow{2}{*}{5} & Sep 16 & Susp. Material, Plankton & Sep 16 & Susp. Material, Plankton\\
 & Sep 22 & Vegetation & Sep 22 & Vegetation
\bottomrule
\end{tabular}
\label{Probenahmetermine}
\end{table}




\begin{table}[htb]{\textwidth}
\caption{Eckdaten für die Standorte entlang des Salzgradienten}
\begin{tabular}{lllll}
\toprule
 Location				& Date		& Mean Depth	& PSU 			& Salinity Range   \\
 						& 			& ( +M /-M )	& 				& (Venice System 1959)\\
\midrule
Geltinger Bucht 		& June 25	& 1,09 / 0,55	& 9,67 - 10,15 	& $ \alpha $ -mesohaline\\ 
Orther Bucht			& June 26 	& 0,66 / -		& 10,34 - 11,49 & $ \alpha $ -mesohaline\\
Salzhaff 				& June 18 	& 0,86/ 0,30	& 7,80 - 8,30  	& $ \alpha $ -mesohaline\\
\midrule
Vitter Bodden			& July 03 	& 0,83/ 0,62	& 8,02 - 8,28  	& $ \beta $ -mesohaline\\
Griebener Bucht			& July 05 	& 0,87/ 0,65	& 8,11 - 8,86	& $ \beta $ -mesohaline\\
\midrule
Spandowerhagener Wiek 	& July 02 	& 0,84/ -		& 2,70 - 2,76  	& $ \alpha $ -oligohaline\\
\bottomrule
\end{tabular}
\label{Salztabelle}
\end{table}


