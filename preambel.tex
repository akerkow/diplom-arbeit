\RequirePackage[l2tabu, orthodox]{nag}


\documentclass[a4paper,11pt,ngerman,DIV=12,BCOR=12mm,titlepage,toc=listof,toc=bib]{scrartcl}

% Tabellen schöner machen:
\usepackage{booktabs}
\usepackage{multicol}
\usepackage{multirow}

\usepackage{enumitem} %Eingerückte Aufzählungen gestalten:

\usepackage{cmap} % to make the PDF files "searchable and copyable" in pdf viewer

\usepackage[T1]{fontenc}
\usepackage[utf8]{inputenc}
\usepackage[ngerman]{babel}
\usepackage[babel,german=quotes]{csquotes}
\usepackage{bibgerm}
\usepackage[comma]{natbib}
\usepackage[font=small,labelfont=bf,format=hang]{caption} % viele Formatierungsmöglichkeiten für die Bildunterschriften
\usepackage[scaled]{beramono} % Monospace Font im Dokument
\usepackage{libertine}
\renewcommand*\familydefault{\sfdefault}  %% Only if the base font of the document is to be sans serif


\usepackage{microtype}          % echter Blocksatz
\usepackage{fixltx2e}           % Verbessert einige Kernkompetenzen von LaTeX2e
\usepackage{ellipsis}           % Korrigiert den Weißraum um Auslassungspunkte
\usepackage{color}
\usepackage{graphicx}           % Paket zum Einbinden von Grafiken:
\usepackage{caption3}
\usepackage{subcaption}			% zum Einbinden von Grafiken nebeneinander!!!
\usepackage{epstopdf}
\usepackage{placeins}			% mit \FloatBarrier maximale Barriere zum Wegrutschen von Bildern setzen


\usepackage[thinqspace, textstyle]{SIunits} %Syntax: \unit{0}{\kelvin} sind \unit{273}{\celsius} oder z.b. {\watt\per\square\meter}, oder \unit{12}{\kilo\square\meter} für Quadratkilometer

\usepackage{textcomp} 			% für geografische Koordinaten nach: Zahl\textdegree\ zahl\textquotesingle\ zahl\dq\ himmelsrichtung 

%\usepackage{amsmath} % Auskommentieren, wenn mathematische Formeln gesetzt werden sollen.
\usepackage{pifont,mathptmx,charter,courier}
\usepackage[scaled]{helvet}

\usepackage{mathrsfs,amssymb}
\usepackage[intlimits]{empheq}
\usepackage{theorem}
%\usepackage{parskip}
\tolerance=2000


% Support for PDF inclusion
\usepackage[final]{pdfpages}

\usepackage{setspace}		% Paket für halben Zeilenabstand
%\doublespacing			% doppelter Zeilenabstand
\onehalfspacing			% Zeilenabstand 1,5


%%% Hyperref-Paket
\usepackage[
	colorlinks,
	pdfpagelabels,
	pdfstartview = FitH,
	bookmarksopen = true,
	bookmarksnumbered = true,
	linkcolor = black,
	plainpages = false,
	hypertexnames = false,
	citecolor = black,
	urlcolor=black,
	pdftitle={Titel der Arbeit}, %%% Titel eintragen, erscheint automatisch in den PDF Eigenschaften
	pdfauthor={Antje Kerkow} %%% Autor eintragen, erscheint automatisch in den PDF Eigenschaften
]{hyperref}

%%% Seitenlayout mit dem KOMA-Paket
\usepackage[%
   % headtopline,
   % plainheadtopline,
    headsepline,
   % plainheadsepline,
    footsepline,
   % plainfootsepline,
   % footbotline,
   % plainfootbotline,
   % ilines,
   % clines,
   % olines,
   automark,
   % autooneside,% ignore optional argument in automark at oneside
   komastyle,
%    standardstyle,
   % markuppercase,
   % markusedcase,
%   nouppercase,
]{scrpage2}





%%%%%%%%%%%%%%%%%%%%%%%%%%%%%%%%%%%%%%%%%%%%%%%%%%%%%
% Schusterjungen und Hurenkinder vermeiden
\clubpenalty = 10000
\widowpenalty = 10000
\displaywidowpenalty = 10000
